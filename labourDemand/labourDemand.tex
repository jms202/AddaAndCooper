\section{Chapter 9: dynamics of employment adjustment}

\subsection{Convex adjustment costs} \label{sub:demandconvex}

This section provides solutions to parts (1) and (3) of Exercise 9.2 on p224 \citet{adda2003dynamic} of The value function we consider is equation (9.9):
%
\begin{equation}
	V(A, e_{-1}) = \max_{h,e} \left\{ R(A,e,h) - \omega(e,h) - C(e,e_{-1}) + \beta EV(A', e) \right\}
\end{equation}
%
where \(R(\cdot)\) is revenue, a function of productivity, \(A\), number of employees, \(e\) and hours worked, \(h\). \(\omega(\cdot\) is the cost of hiring \(e\) workers to work \(h\) hours. \(C(\cdot)\) is the cost of adjusting the number of workers from \(e_{-1}\) in the previous period to \(e\) this period.

The files \texttt{labourDemand.py} and \texttt{estimateLabourDemand.py} solve and estimate parameters for this model in line with parts (1) and (3) of Exercise 9.2.

The functional forms used are as suggested:
%
\begin{align}
 R(A,e,h) &=  A(eh) \\
 \omega(e,h) &= we\left[w_0 + h + w_1(h - 40) + w_2(h - 40)^2\right] \\
 C(e,e_{-1}) &= \frac{\eta}{2}\frac{(e - e_{-1})^2}{e_{-1}}
\end{align}
%
With the estimation routine, an important thing to note again is that---because of the simulated, discontinuous objective---the minimisation algorthim struggles to find the true minimum.

\subsection{Piecewise linear adjustment costs}

This section considers Exercise 9.3 on p225 of \citet{adda2003dynamic}. It is an identical problem to the case with convex adjustment costs above except for the specification of the adjustment costs function.
%
\begin{equation}
  C(e,e_{-1})=\begin{cases}
    \gamma^{+}\Delta e & \text{if $\Delta e > 0$}\\
    \gamma^{-}\Delta e & \text{otherwise}
  \end{cases}
\end{equation}
%
The file \texttt{labourDemandPiecewise.py} solves and estimates parameters for this model.

\subsection{Nonconvex adjustment costs}

The code \texttt{labourDemandNonConvex.py} solves and simulates an investment model with nonconvex adjustment costs as described in Section 9.4.2 and Exercise 9.4 of \citet{adda2003dynamic}. The value function is given by:
%
\begin{equation}
	V(A, e_{-1}) = \max \left\{ V^a(A, e_{-1}), V^n(A, e_{-1}) \right\}
\end{equation}
%
where the \(a\) and \(n\) superscripts signify adjusting employment and not adjusting respectively, and other variables are as above. The choice-specific value functions are given by:
%
\begin{align}
	V^a(A, e_{-1}) &= \max_{h,e} \left\{ R(A,e,h) - \omega(e,h) - F + \beta EV(A', e) \right\} \\
	V^n(A, e_{-1}) &= \max_{h} \left\{ R(A,e_{-1},h) - \omega(e_{-1},h) + \beta EV(A', e_{-1}) \right\}
\end{align}
%
where \(F\) is the fixed cost of adjusting the employment input.